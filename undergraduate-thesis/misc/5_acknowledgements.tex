%%
% The BIThesis Template for Bachelor Graduation Thesis
%
% 北京理工大学毕业设计(论文)致谢 —— 使用 XeLaTeX 编译
%
% Copyright 2020-2022 BITNP
%
% This work may be distributed and/or modified under the
% conditions of the LaTeX Project Public License, either version 1.3
% of this license or (at your option) any later version.
% The latest version of this license is in
%   http://www.latex-project.org/lppl.txt
% and version 1.3 or later is part of all distributions of LaTeX
% version 2005/12/01 or later.
%
% This work has the LPPL maintenance status `maintained'.
%
% The Current Maintainer of this work is Feng Kaiyu.
%
% Compile with: xelatex -> biber -> xelatex -> xelatex

% 致谢部分尽量不使用 \subsection 二级标题,只使用 \section 一级标题
\begin{acknowledgements}
首先,我要由衷地感谢我的导师张艳老师和逄金辉老师。在毕业设计的过程中,她们给予了我诸多指导意见,无私地利用自己的休息时间阅读和批阅我的论文,才使得我能够完成现在的论文终稿。她们的专业知识和辛勤付出对我的成长起到了至关重要的作用。

在本科四年的学习中,我还要感谢许多老师、家人和朋友同学。首先是在数字图像处理课堂上的邸慧军老师,他用心讲解将我引入了计算机视觉的大门。他的课堂教学充满了激情和启发,让我对数字图像处理产生了浓厚的兴趣。同时,我要特别感谢自动化学院的高琪老师。高老师不仅在学业上提供了宝贵的建议和指导,还对我的学习方法和人生规划给予了真挚的关怀。他教会了我如何进行高效的学习,如何在追求知识的同时注重自身的成长和发展。高老师的言传身教对我产生了深远的影响,使我更加自信和坚定地追求自己的目标。

此外,我要感谢清华大学智能产业研究院的赵昊和石永亮老师。在我的研究过程中,他们给予了我深度指导和发挥自己能力的平台。赵老师和石老师具有丰富的研究经验和深厚的学术造诣,他们在研究方向和方法上给予了我宝贵的指导。他们不仅分享了他们的专业知识,还鼓励我独立思考和探索,为我提供了展示自己能力的机会。在他们的引领下,我逐渐成长为一个有自主思考和解决问题能力的研究者。

除了老师们,我要特别感谢我的家人,在我进行科研的过程中给予了无条件的支持和理解。他们始终鼓励我追求知识,激励我克服困难,并在我最需要的时候给予我无私的支持。无论是在研究中遇到困难,还是在学业压力下感到疲惫,他们总是在我身边给予我鼓励和支持的力量。他们的关爱和支持是我不断前行的动力,我感激他们一直以来对我坚定追求知识的支持。

此外,我的朋友和同学们也给了我宝贵的支持和帮助。他们是我成长道路上不可或缺的伙伴和知音。每当我遇到学术困惑或个人疑虑时,他们总是倾听我的心声,并给予我中肯的建议和意见。我们相互鼓励、相互支持,一同成长,共同追逐梦想。他们的友谊和支持让我感到无比幸运和温暖。

在此,我想向我的家人和朋友们表达我最深的谢意。感谢你们的无私支持、关怀和鼓励,使我能够坚持不懈地追求知识和追逐梦想。你们的陪伴和支持是我最宝贵的财富,我会倍加珍惜,并将这份爱与支持传递下去,回报于社会和他人。

最后,我要再次感谢所有给予我支持和帮助的人。是你们的支持和鼓励,让我能够顺利完成这篇论文。

\end{acknowledgements}
