%%
% The BIThesis Template for Bachelor Graduation Thesis
%
% 北京理工大学毕业设计(论文)结论 —— 使用 XeLaTeX 编译
%
% Copyright 2020-2022 BITNP
%
% This work may be distributed and/or modified under the
% conditions of the LaTeX Project Public License, either version 1.3
% of this license or (at your option) any later version.
% The latest version of this license is in
%   http://www.latex-project.org/lppl.txt
% and version 1.3 or later is part of all distributions of LaTeX
% version 2005/12/01 or later.
%
% This work has the LPPL maintenance status `maintained'.
%
% The Current Maintainer of this work is Feng Kaiyu.
%
% Compile with: xelatex -> biber -> xelatex -> xelatex

\begin{conclusion}
  % 结论部分尽量不使用 \subsection 二级标题,只使用 \section 一级标题

% \section{总结}
  % 这里插入一个参考文献,仅作参考
  本文针对实现真实道路驾驶数据中场景地图三维重建的问题,提出了一种基于混合隐式场景表示的方法。通过分析现有视觉三维重建方法的局限性,本文能够融合多元传感器信息输入(包括RGB图片、LiDAR深度感知信息和GPS位置信息等),重建一个动态混合神经隐式场景模型,可以进行真实感图片渲染、精确深度信息预测。为了解决真实自动驾驶场景中的退化输入数据问题,本文提出了多传感器时间同步和雾天受散射效应影响较大两个问题的解决方案。此外,本文还介绍了基于混合神经隐式场构建的动态场景图方法,实现对动态物体的理解和编辑。

本文的主要工作总结如下:
\begin{enumerate}
    \item 本文提出了一种在理想RGB-D数据输入的情况下,基于多元传感信息建模混合距离-辐射场,解决了现有方法中普遍存在的混合隐式场内在误差和距离-深度二义性,可以在场景外观颜色渲染和场景几何建模等任务上超过现有方法;
    \item 从异步RGBD序列训练城市规模的神经辐射场,通过利用隐式轨迹先验,解决了现实应用中遇到的传感器不同步问题,实现在真实大规模场景中融合多元传感信息的地图表示学习。此外,本文首次提出从带雾图像中学习神经隐式场的方法,通过对悬浮粒子权重重新分配,允许灵活的散射效应控制,通过最小化协方差损失,最大化保留场景中的视角相关颜色;
    \item 从RGB-D真实数据中学习混合隐式场景图,使得静态场景和动态物体表示成功解耦,实现了对复杂动态场景的理解和编辑。
\end{enumerate}

本文在公开数据集和各种退化条件下,在仿真、真实场景中进行详尽的对比实验,验证了本课题所提出方法显著优于现有方法。综合以上研究成果,本文为实现从真实多元传感的自动驾驶数据中构建隐式自动驾驶仿真环境,从而实质性推动自动驾驶Real2Sim2Real的技术发展做出了重要贡献。

\clearpage
% \section{未来展望}
\textbf{未来展望:}
本文通过对现有神经隐式场方法和真实世界中的传感数据问题的分析,提出了面向Real2Sim仿真场景重建的地图表征方法。然而,本文的研究内容仍然存在一些不足,具有一定的改进空间:

\textit{提高混合隐式场景表示的效率和可扩展性}:尽管所提出的混合隐式表示已经显示出不错的结果,但在效率和可扩展方面仍有改进的空间,特别是对于大规模和复杂的城市环境。未来的研究可以集中在开发更高效的算法,以处理更大的场景,并提高混合隐式表示的计算效率。

\textit{适应各种天气和照明条件}:目前的工作解决了在雾天气条件下重建3D场景的问题。然而,在未来的研究中,可以研究其他具有挑战性的天气和照明条件,如大雨、雪和夜间,以提高所提出的方法在更广泛环境中的适用性。

\textit{将语义信息纳入场景表示}:目前的工作重点是场景表示的几何和外观方面。将语义信息(如对象分类和场景理解)纳入场景表示中,可以进一步增强所提出方法的能力,并在自动驾驶中实现更高级的应用,例如预测行人和其他车辆的行为。

\textit{建模人体等复杂肢体动作}:本文方法中,主要解决了动态场景中的车辆等物体的刚体运动。然而,在真实数据中,人体的更为复杂的非刚体运动同样广泛存在。在未来,可以通过引入人体动作模型对这些复杂动作进行进一步建模,从而进一步推动三维重建技术的发展。

  % \textcolor{blue}{结论作为毕业设计(论文)正文的最后部分单独排写,但不加章号。结论是对整个论文主要结果的总结。在结论中应明确指出本研究的创新点,对其应用前景和社会、经济价值等加以预测和评价,并指出今后进一步在本研究方向进行研究工作的展望与设想。结论部分的撰写应简明扼要,突出创新性。阅后删除此段。}

  % \textcolor{blue}{结论正文样式与文章正文相同:宋体、小四;行距:22 磅;间距段前段后均为 0 行。阅后删除此段。}
\end{conclusion}

