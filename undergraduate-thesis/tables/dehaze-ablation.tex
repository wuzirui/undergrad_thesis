% Please add the following required packages to your document preamble:
% \usepackage{graphicx}
\begin{table}[ht]
\centering
\resizebox{0.6\textwidth}{!}{%
\begin{tabular}{llll}
\hline
方法             & PSNR $\uparrow$ & SSIM $\uparrow$ & LPIPS $\downarrow$ \\ \hline
Ours           & \textbf{19.71}  & \textbf{0.72}   & \textbf{0.28}      \\
Ours w/o 视角相关性 & 16.26           & 0.66            & 0.32               \\
Ours w/o 协方差损失 & 14.43           & 0.63            & 0.36               \\
Ours w/o 深度权重  & 12.62           & 0.54            & 0.46               \\
DCP-NeRF       & 8.005           & 0.43            & 0.48               \\ \hline
\end{tabular}%
}
\caption{去雾方法消融实验中渲染颜色相关定量指标评估}
\label{tab:dehaze-ablation-rgb}
\end{table}

% Please add the following required packages to your document preamble:
% \usepackage{graphicx}
\begin{table}[ht]
\centering
\resizebox{\textwidth}{!}{%
\begin{tabular}{llllllll}
\hline
方法 &
  RMSE $\downarrow$ &
  RMSE log $\downarrow$ &
  Abs. Rel $\downarrow$ &
  Sq. Rel $\downarrow$ &
  $\delta_1\ \uparrow$ &
  $\delta_2\ \uparrow$ &
  $\delta_3\ \uparrow$ \\ \hline
Ours           & 2.71    & 0.0577 & 0.0249          & 0.1519  & 0.9966 & 0.9993 & 0.9995 \\
Ours w/o 视角相关性 & 1.76    & 0.0445 & \textbf{0.0137} & 0.0734  & 0.9981 & 0.9990 & 0.9993 \\
Ours w/o 协方差损失 &
  \textbf{1.53} &
  \textbf{0.0309} &
  0.0139 &
  \textbf{0.0563} &
  \textbf{0.9991} &
  \textbf{0.9994} &
  \textbf{0.9996} \\
Ours w/o 深度权重  & 32.3166 & 1.0395 & 0.6092          & 20.2141 & 0.0583 & 0.06   & 0.122  \\
DCP-NeRF       & 40.2796 & 0.6245 & 0.6236          & 40.0531 & 0.1992 & 0.4275 & 0.6877 \\ \hline
\end{tabular}%
}
\caption{去雾方法消融实验中深度预测相关定量指标评估}
\label{tab:dehaze-ablation-depth}
\end{table}

% Please add the following required packages to your document preamble:
% \usepackage{graphicx}
\begin{table*}[ht]
\resizebox{\textwidth}{!}{%
\begin{tabular}{cc|cccccccccc}
\hline
\textbf{thresh} &
  \textbf{p} &
  \textbf{PSNR} &
  \textbf{SSIM} &
  \textbf{LPIPS} &
  \textbf{RMSE} &
  \textbf{RMSE log} &
  \textbf{Sq. Rel} &
  \textbf{Abs. Rel} &
  \textbf{$\delta_1$} &
  \textbf{$\delta_2$} &
  \textbf{$\delta_3$} \\ \hline
100 & 4           & 19.626          & 0.7211 & 0.2835 & 2.3631  & 0.0517 & 0.1177 & 0.0240 & 0.9973 & 0.9994 & \textbf{0.9996} \\
200 & 4           & 19.657          & 0.7206 & 0.2838 & 2.6717  & 0.0567 & 0.1472 & 0.0249 & 0.9967 & 0.9994 & 0.9995          \\
300 & 4           & \textbf{19.664} & 0.7208 & 0.2836 & 2.7275  & 0.0584 & 0.1531 & 0.0248 & 0.9964 & 0.9993 & 0.9994          \\
400 & 4           & 19.658          & 0.7210 & 0.2828 & 2.6919  & 0.0611 & 0.1503 & 0.0240 & 0.9964 & 0.9991 & 0.9993          \\
256 & $\sqrt{2}$  & 16.003          & 0.6662 & 0.3298 & 11.6805 & 0.2512 & 2.5377 & 0.1930 & 0.4199 & 0.9672 & 0.9970          \\
256 & $\sqrt{4}$  & 18.818          & 0.7159 & 0.2835 & 3.0478  & 0.0605 & 0.1889 & 0.0411 & 0.9864 & 0.9990 & 0.9995          \\
256 &
  $\sqrt{8}$ &
  19.566 &
  \textbf{0.7230} &
  \textbf{0.2786} &
  \textbf{1.9347} &
  \textbf{0.0378} &
  \textbf{0.0798} &
  \textbf{0.0190} &
  \textbf{0.9979} &
  \textbf{0.9994} &
  0.9995 \\
256 & $\sqrt{32}$ & 19.645          & 0.7195 & 0.2844 & 3.0540  & 0.0878 & 0.1935 & 0.0265 & 0.9955 & 0.9983 & 0.9985          \\ \hline
\end{tabular}%
}
\caption{在参数thresh和p上选择不同组合对实验结果的影响}
\label{tab:ablation-config}
\end{table*}