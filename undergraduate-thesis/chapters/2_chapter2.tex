\chapter{文献综述}
在本章中,我们将对国内外现有相关研究工作现状进行阐述。在\ref{sec: related-work implicit scene representation learning}节中,我们将介绍以神经辐射场\cite{mildenhall_nerf_2020}为代表的使用隐式函数进行场景表示学习的主流方法。纵观神经隐式场发展的历程,人们用来学习场景外观的地图表示方法经历了从纯隐式的多层感知机地图到半隐式的体素地图、低维张量地图的变革,在该节中我们将详细介绍这一变换的过程和内在原因。在\ref{sec: related-work realistic rendering}一节中,我们将介绍人们基于神经辐射场上为了使渲染更具真实感所提出的一系列方法,包括引入无界场景压缩、前后景分割等。在\ref{sec: related-work density-distance fields}一节中,我们将介绍辐射-距离混合场景表示的方法。在该节中,我们将首先分析单纯使用辐射场同时表示场景几何和外观所带来的二义性问题,接着从采样策略、体渲染方法、多元传感信息融合以及模型优化几个角度分别介绍相关工作所使用的方法。

\section{神经隐式场景表示学习}
\label{sec: related-work implicit scene representation learning}

神经隐式场景表示,即是将三维场景的几何、外观、语义等信息抽象为一个隐式表示的函数,通过神经网络的回归能力来拟合这些隐函数。在本节中,我们将介绍使用以神经辐射场和神经符号距离场为主的神经隐式表征进行场景表示学习的主要方法。在\ref{sec: related-work MLP-based neural implicit representations}节,我们将通过神经辐射场和DeepSDF介绍使用多层感知机表示场景隐式函数的基本概念,在\ref{sec: related-work grid-based implicit representations}和\ref{sec: related-work multi-plane-based implicit representations}两节中介绍神经隐式场景表示的主干模型的改进。


\subsection{基于多层感知机的场景表示}
\label{sec: related-work MLP-based neural implicit representations}

在2020年,Ben Mildenhall等人在ECCV 2020会议上发表了神经辐射场\cite{mildenhall_nerf_2020}(NeRF)用来解决新视角合成任务(如图\ref{fig:related-work novel-view-synthesis-task}所示)。新视角合成任务输入为已知位姿的多视角图片$\{{I}_i\}^N$,希望在该场景的任意给定新视角下,渲染一张具有真实感的图片${I}_\text{target}$。所谓神经辐射场,即是在假设三维空间中任意一个三维点均在向外辐射光线并阻挡其他光线的传播这一前提下,使用一个隐式神经网络建模每个三维点的辐射量。

\begin{figure}[h]
    \centering
    \includegraphics[width=\textwidth]{undergraduate-thesis/images/related-work/NeRF setting.pdf}
    \caption{使用神经辐射场进行新视角合成的基本流程}
    \label{fig:related-work novel-view-synthesis-task}
\end{figure}

具体来说,辐射场可以表达为一个输入为三维欧氏空间下坐标$(x,y,z)$和观察角度$(\theta,\phi)$,输出为RGB颜色值$(R,G,B)$和体密度$\sigma$的函数,如公式\ref{eq:related-work radiance field}所示,其中$(x,y,z)\in\mathbb{R}^3, (\theta,\phi)\in[0, 2\pi]^2, \mathbf{c}=(R,G,B)\in[0,255]^3, \sigma\in\mathbb{R}^+$。RGB色值为空间点向外辐射的光的颜色,为了建模高光等物理光照现象,这个值应随观察角度的变化而变化,即$f_{\{R,G,B\}}:f_{\{R,G,B\}}(x,y,z,\theta,\phi)$,相反,体密度$\sigma$只建模三维点的物理密度,是空间点坐标的函数,与观察视角无关,即$\sigma:\sigma(x,y,z)$。根据万能近似定理\cite{hornik_multilayer_1989}(Universal Approximation Theorem),NeRF使用一个多层感知机来建模这种隐式的函数关系,即神经辐射场。
\begin{equation}
    f_\text{NeRF}: (x,y,z,\theta,\phi)\to(R,G,B,\sigma).
    \label{eq:related-work radiance field}
\end{equation}

通过累计一条光线上三维点的辐射值,我们便可以实现在任意观察视角下光线的渲染。图\ref{fig:related-work NeRF pipeline}展示了NeRF方法的总体流程:
\begin{enumerate}
    \item [a] 在观察视角发射一条光线,并在其上采样若干个三维空间点,将他们的三维欧式坐标和观察角度组成一个5元组张量$(x_i,y_i,z_i,\theta,\phi)$;
    \item [b] 通过神经辐射场,得到采样点的隐函数值$(R_i,G_i,B_i,\sigma_i)$;
    \item [c] 使用体积渲染公式,计算整条光线的渲染颜色值$\hat{c}$;
    \item [d] 将渲染值$\hat{c}$与真实观测值${c}_{gt}$比较,计算损失函数,通过反向传播优化神经网络。
    \begin{align}
    c(\mathbf{o},\mathbf{d}) &= \int_0^\infty \sigma(\mathbf{o}+t\mathbf{d})T(t)c(\mathbf{o}+t\mathbf{d},\mathbf{d})\text{d}t,\label{eq: related-work volume rendering}
    \\
    T(t) &= \exp(-\int_0^t\sigma(\mathbf{o}+t\mathbf{d})\text{d}t).\label{eq: related-work transmittance}
    \end{align}
\end{enumerate}


体渲染公式如公式\ref{eq: related-work volume rendering}所示,其中$T(t)$为累计的透光度,如公式\ref{eq: related-work transmittance}所示,通过体渲染,我们可以将三维辐射场的场景信息投影到二维的成像平面上,进行真实感的渲染。


\begin{figure}[t]
    \centering
    \includegraphics[width=\textwidth]{undergraduate-thesis/images/related-work/NeRF method.pdf}
    \caption{使用神经辐射场进行训练和渲染的总体流程}
    \label{fig:related-work NeRF pipeline}
\end{figure}

\begin{figure}[ht]
    \centering
    \includegraphics[width=0.7\textwidth]{undergraduate-thesis/images/related-work/SDF concept.pdf}
    \caption{符号距离场示例。图中(a)展示了由SDF函数所隐式确定的表面;(b)为符号距离场在二维的分布,蓝色区域表示SDF<0的区域,红色区域表示SDF>0的区域;(c)由三维符号距离场所确定的几何信息进一步导出的物体模型。}
    \label{fig:related-work SDF concept}
\end{figure}

除了辐射场中的函数关系,基于多层感知机的神经隐式表达也可以被用于其他任务的实现上\cite{park_deepsdf_2019, mescheder_occupancy_2019, shim_snerl_2023, zhi_-place_2021}。2019年Park等人使用多层感知机来拟合场景的三维符号距离场上。符号距离场(Signed Distance Fields, SDF)描述了场景中三维点到物体表面距离,用一个隐式函数$f_{SDF}$表示,如图\ref{fig:related-work SDF concept}所示,有向距离值的绝对值表示某三维点$\mathbf{x}(x,y,z)$到其最近表面的距离,其符号表示三维点在表面内/外,当三维点位于一个封闭曲面内部,则其$f_{SDF}(\mathbf{x}) < 0$,在表面外时$f_{SDF}(\mathbf{x}) > 0$。因此$f_{SDF}(\mathbf{x})$的零值面即为所描述的三维曲面。在DeepSDF中,作者使用一个多层感知机来拟合这一隐式函数。

\begin{figure}[ht]
    \centering
    \includegraphics[width=0.8\textwidth]{undergraduate-thesis/images/related-work/deepSDF.pdf}
    \caption{DeepSDF网络结构的两种形式:左侧为一般情形下直接拟合物体的SDF隐函数;右侧是带有类别编码的SDF函数。}
    \label{fig:related-work DeepSDF}
\end{figure}

不同于神经辐射场直接使用多层感知机拟合场景隐函数,DeepSDF引入类别编码,为一类具有相似形状的物体学习类别表征编码,与隐函数输入同时输入网络。这样做可以大大缩减学习同类物体SDF网络的时间,网络结构如图\ref{fig:related-work DeepSDF}所示。



\subsection{基于体素网格的场景表示}
\label{sec: related-work grid-based implicit representations}
虽然多层感知机作为一个万能函数拟合器在理论上可以无限接近场景的隐式表征函数,但这样的表示主干网络在实践中存在若干问题:
\begin{enumerate}
    \item 多层感知机的训练需要较大计算量,即使训练一个较小物体或场景的隐式函数也往往需要1$\sim$2天的训练时间才能使网络收敛,前向传播的耗时也不能满足计算机图形学实时渲染的需求;
    \item 多层感知机的训练存在遗忘问题。在一组数据上训练产生的梯度往往会影响其他数据上已经收敛的模型权重,这导致以多层感知机为主干网络的模型往往不能泛化到更大的场景上(如完整的室内场景等)。
\end{enumerate}

基于这样的观察,研究者尝试使用基于体素的场景地图主干网络。基于体素网格的场景表示可以采用一种更加直观的方式来表示3D场景,即将3D场景划分为一个网格,并在每个网格内记录它的辐射、法向量等信息。与基于多层感知机的场景表示相比,在训练和前向传播方面,基于体素网格的方法具训练时间短、效果更好等特点。

\subsubsection{传统密集体素网格地图}
早期使用体素网格地图表示的方法\cite{fridovich-keil_plenoxels_2022, kondo_vaxnerf_2021, yu_monosdf_2022}通常将三维场景离散化为一个三维空间网格,其中的每个体素(Voxel)的每个节点上存储该点上的隐式函数值。当需要查询任意一个连续空间中的三维点的隐函数值时,只需查询其周围的八个相邻节点,并使用三线性插值计算目标点的隐函数值。

我们用$\mathcal{G}_\theta$来表示一个分辨率为$R_H\times R_W\times R_D$的体素网格,则使用体素网格来作为隐式场骨干模型的方法可以写作:
\begin{equation}
    \hat{s} = \mathtt{interp}(\mathbf{x}, \mathcal{G}_\theta),
\end{equation}
其中$\hat{s}$为网络预测的隐式函数值,$\mathtt{interp}$表示三线性插值。

然而,将三维网格应用于取代原始NeRF中多层感知机的一个必然结果就是对于输入数据的限制。具体而言,原始NeRF使用物体的三维位置坐标和观察角度作为输入来建模物体的Lambertian和非Lambertian效应(即漫反射、高光等与视角相关的效应),然而使用三维网格来建模物体的辐射量则限制了输入维度仅能为三维。这种限制在某些场景下可能会对建模效果产生一定影响。

\begin{figure}[ht]
    \centering
    \includegraphics[width=\textwidth]{undergraduate-thesis/images/related-work/plenoxels.pdf}
    \caption{Plenoxels 方法流程图}
    \label{fig:related-work plenoxels pipeline}
\end{figure}

为了弥补这样的缺陷,Plenoxels\cite{fridovich-keil_plenoxels_2022}中提出在三维网格中同时使用球面高斯谐波(Spherical Harmonics, SH)函数来建模场景的视角相关特性,如图\ref{fig:related-work plenoxels pipeline}所示。SH函数可以表示为一组具有不同频率和幅度的球面函数,其中每个函数代表在特定方向上的照射光线所对应的光照贡献。由于球面高斯谐波函数是以角度为参数的球面函数,因此能够很好地表达几何结构与视角之间的关系。通过计算遍历每个像素所对应的差分方程,可以得到具有视角相关效果的图像。从实践上,即是使三维网格在存储场景的辐射颜色和体密度之外,同时存储球面高斯谐波系数组,在查询时通过将插值得到的球面谐波系数带入谐波方程,来计算视角相关颜色。 

\begin{figure}[h]
    \centering
    \includegraphics[width=.5\textwidth]{undergraduate-thesis/images/related-work/plenoxels-result.pdf}
    \caption{Plenoxels和原始NeRF的性能对比}
    \label{fig:related-work plenoxels-result}
\end{figure}

图\ref{fig:related-work plenoxels-result}中展示了Plenoxels方法和NeRF方法的性能对比,其中横轴为训练时间,纵轴为PSNR指标(越高越好)。由图可知,使用密集体素网格可以为隐式场训练带来成百倍的速度提升,且在最终渲染的效果上也要优于基于多层感知机的方法。

除此以外,当使用密集体素网格重建三维隐式表面时,由于体素网格不能提供多层感知机中的平滑特性(我们将在\ref{sec: related-work density-distance fields}节中讨论该特性),重建的三维表面通常会出现嘈杂的现象(图\ref{fig:related-work monosdf result})。

\subsubsection{隐式体素参数网格地图}
虽然上述的密集体素网格在渲染速度和效果上均有较大的提升,然而在面对更大规模的场景时,仅使用密集网格存储隐函数信息的网络表达能力并不足够,且使用密集网格进行表面重建也会遇到嘈杂的结果,为此,人们提出在网格中存储空间特征而非显式的隐函数值\cite{liu_neural_2021, takikawa_neural_2021, yu_monosdf_2022, huang_di-fusion_2021, peng_convolutional_2020},并使用一个后续的多层感知机解码器来计算最终隐函数值。

记分辨率为$R\times R\times R$的参数网格为$\Phi_\theta$, 多层感知机解码器为$f_\theta$,则使用参数网格来计算隐式函数值可以表示为:
\begin{equation}
    \hat{s} = f_\theta(\gamma(\mathbf{x}), \mathtt{interp}(\mathbf{x}, \Phi_\theta),
\end{equation}
其中$\gamma(\mathbf{x})$为$\mathbf{x}$的位置编码,我们将在\ref{sec: related-work positional encoding}一节讨论位置编码。

在此指上,为了进一步提高网络在场景细节上的建模能力,人们引入了多分辨率的体素参数网格地图\cite{yu_monosdf_2022},即使用多个不同分辨率的网格$\{\Phi^l_\theta\}^L_{l=1}$代替单一参数网格$\Phi_\theta$。记所有不同分辨率的网格中最小的网格($l=1$)的分辨率为$R_{\min}^3$,最大的网格($l=L$)分辨率为$R_{\max}^3$,则第$l$大的网格的分辨率可以表示为:
\begin{equation}
    R_l:=R_{\min}\cdot b^l, \quad l=\mathtt{floor}(\exp(\frac{1}{L - 1}(\ln R_{\max} - \ln R_{\min}))),
\end{equation}
进一步,查询多分辨体素参数网格的过程可以表示为:
\begin{equation}
    \hat{s} = f_\theta(\gamma(\mathbf{x}), \mathtt{concat}\{\mathtt{interp}(\mathbf{x}, \Phi_\theta^l)\}_{l=1}^L),
\end{equation}
$\mathtt{floor}$为下取整函数,$\mathtt{concat}$表示参数拼接。

\begin{figure}[ht]
    \centering
    \includegraphics[width=\textwidth]{undergraduate-thesis/images/related-work/monosdf-result.pdf}
    \caption{使用多层感知机、密集体素网格、体素特征网格和多分辨率体素特征网格进行场景SDF学习的结果}
    \label{fig:related-work monosdf result}
\end{figure}

我们在图\ref{fig:related-work monosdf result}中比较神经隐式表面表示的不同设计选择,我们观察到密集的 SDF 网格会导致嘈杂的重建结果,而 MLP 和单分辨率体素网格改进了结果,但重建的几何图形往往过于平滑而缺少细节,使用多分辨率体素网格可获得较好的结果。

\subsubsection{基于哈希编码的体素参数网格地图}
由于体素参数地图方法在存储大规模场景时,需要使用高分辨率的参数网格来获得足够的细节并保证精度,这导致了空间复杂度高达$\mathcal{O}(N^3)$。然而,在三维场景中,人们通常只关注表面,其所需要的数据量只随场景大小呈平方的空间复杂度($\mathcal{O}(N^2)$)。为了解决这一问题,M\"uller等人提出了一种基于哈希编码的体素参数网格地图——InstantNGP\cite{muller_instant_2022}。 InstantNGP使用哈希表来存储隐式函数值,只需控制哈希表的大小随场景大小平方增长即可。

\begin{figure}[ht]
    \centering
    \includegraphics[width=\textwidth]{undergraduate-thesis/images/related-work/instantngp.pdf}
    \caption{InstantNGP 方法流程图}
    \label{fig:related-work instantngp}
\end{figure}

如图\ref{fig:related-work instantngp}所示,InstantNGP沿用了多分辨率体素网格的基本框架,在查询任意一点的三维场景隐函数值时,首先需要查询其在各个层次的网格上的八个相邻节点,接着将节点编号作为索引查询哈希表,将查询结果经三线性插值之后拼接形成三维点特征。我们可以将这个过程表示为:
\begin{equation}
    \hat{s} = f_\theta(\gamma(\mathbf{x}), \mathtt{concat}\{\mathtt{interp}(h^l(\mathbf{x}), \Phi_\theta)\}_{l=1}^L),
\end{equation}
其中$h^l(\mathbf{x})$为哈希函数,在InstantNGP中,每个级别的参数网格使用不同的哈希函数:
\begin{equation}
    h^l(\mathbf{x}) = \left(\bigoplus_{i}x_i\pi_i^l\right) \text{mod} T,
\end{equation}
$T, \pi^l$为随机大质数。

\subsection{低维张量地图表示}
\label{sec: related-work multi-plane-based implicit representations}
\subsubsection{基于张量分解的向量-矩阵地图表示}
虽然InstantNGP基于哈希表的存储数据结构可以有效地缓解基于体素网格的场景表示中高空间复杂性的问题,然而研究者发现现有方法所使用的体素网格中绝大部分体素均属于空白空间,因而网格的使用率实际并不高。Anpei等人\cite{chen_tensorf_2022}的工作TensoRF中利用了一个事实,即特征网格可以自然地被视为 4D 张量,其中它的三个模式对应于网格的 XYZ 轴,第四个模式代表特征通道维度,从而解决了体素网格表示的低效问题,从而产生了一系列简单而有效的方法。TensoRF利用经典张量分解技术(已广泛应用于各个领域的高维数据分析和压缩 \cite{kolda_tensor_2009}),将辐射场的张量分解为多个低秩张量分量,从而获得准确且紧凑的场景表示。

\begin{figure}[h]
    \centering
    \includegraphics[width=\textwidth]{undergraduate-thesis/images/related-work/Tensor Decomp.png}
    \caption{张量分解。左图:CP 分解,它将张量分解为向量外积之和。右图:TensoRF的向量矩阵分解,它将张量分解为向量矩阵外积的总和。}
    \label{fig:related-work tensor-decomp}
\end{figure}

传统的CP张量分解(如式\ref{eq: related-work CP decomp}所示),将三维张量$\mathcal{T}\in\mathbb{R}^{I\times J\times K}$分解为三组秩为一的向量乘积的和。TensoRF将CP分解拓展为向量-矩阵分解模式(如式\ref{eq: related-work TensoRF decomp}所示)。对于每个分解单元,我们将CP-分解中的其中两个分量合并,不使用单独的向量,而是组合每两个模式并用矩阵表示它们,从而允许每个分解单元使用较少数量的组件进行充分参数化。
\begin{equation}
    \mathcal{T} = \sum_{r=1}^R\mathbf{v}_r^1\circ\mathbf{v}_r^2\circ\mathbf{v}_r^3
    \label{eq: related-work CP decomp}
\end{equation}

\begin{equation}
    \mathcal{T} = \sum_{r=1}^R\mathbf{v}_r^X\circ\mathbf{M}_r^{Y,Z}+\mathbf{v}_r^Y\circ\mathbf{M}_r^{X,Z}+\mathbf{v}_r^Z\circ\mathbf{M}_r^{X,Y}
    \label{eq: related-work TensoRF decomp}
\end{equation}

TensorRF (VM)使用一组向量 ($\mathbf{v}$) 和矩阵 ($\mathbf{M}$) 将辐射场建模为张量,它们沿相应的 (XYZ) 轴描述场景,并用于计算可微分光线行进中的体积密度 $\sigma$ 和视角相关颜色 $c$。对于每个着色位置$ \mathbf{x} = (x, y, z)$,我们使用来自向量/矩阵因子的线性/双线性插值结果来有效地计算张量分量的特征。该特征被进一步用于渲染,从而得到一个紧致的神经隐式表示。

\begin{figure}[ht]
    \centering
    \includegraphics[width=\textwidth]{undergraduate-thesis/images/related-work/kplanes-teaser.pdf}
    \caption{$d$ 维空间的平面分解示意图}
    \label{fig:related-work kplane-teaser}
\end{figure}

\subsubsection{多平面地图表示}
受TensoRF启发,研究者进一步将张量分解的概念进一步泛化,仅使用多个互相正交的平面来代替体素网格\cite{fridovich-keil_k-planes_2023, cao_hexplane_2023, reiser_merf_2023, chan_efficient_2022}。其中K-Planes\cite{fridovich-keil_k-planes_2023}将$d$维体素张量使用$k=\frac{d(d-1)}{2}$个平面表示,即$d$个维度的任意两维均用一个平面表示,如图\ref{fig:related-work kplane-teaser}所示。例如,对于静态 3D 场景,这会产生三平面,其中3个平面代表 XY、XZ 和 YZ。而对于动态 4D 场景,这会产生六平面,其中6 个平面,包括三个纯空间平面和三个时空平面 X-t、Y-t 和 Z-t。如果我们希望表示一个 5D 空间,我们可以使用$\frac{5\times4}{2}=10$个平面。


\begin{figure}[h]
    \centering
    \includegraphics[width=\textwidth]{undergraduate-thesis/images/related-work/hadamard-product.png}
    \caption{平面特征的逐元素加法(左)与乘法(右)的比较。在这两种情况下,假设每个平面中只有单个条目为+1,其余部分均为零,通过乘法可以选择单个 3D 点,但通过加法只能产生相交线。这说明乘法可以提高显式模型的表达能力。}
    \label{fig:related-work hadamard-product}
\end{figure}

不同于最早的三平面场景表示\cite{chan_efficient_2022}中使用逐像素的求和来整合各平面上的特征,K-Planes中使用Hadamard乘积来获取空域感知的特征信号(如\ref{fig:related-work hadamard-product}所示),即:
\begin{equation}
    f(\mathbf{q}) = \prod_{c\in C}f(\mathbf{q})_c,
\end{equation}
其中$c\in C$为场景中的平面集合,在每个平面中查询$\mathbf{q}$点处的特征,通过逐像素求乘积得到高维空间中点的特征。

\subsection{小结}
在本节中,我们介绍了神经隐式场出现以来,所使用的场景表示从多层感知机\cite{mildenhall_nerf_2020,park_deepsdf_2019, mescheder_occupancy_2019, shim_snerl_2023, zhi_-place_2021}逐渐发展为体素网格\cite{fridovich-keil_plenoxels_2022,yu_monosdf_2022,muller_instant_2022,jiang_instantavatar_2022}、多平面\cite{chen_tensorf_2022,fridovich-keil_k-planes_2023,cao_hexplane_2023}等场景表示方法。

虽然多平面的方法可以对一个简单场景进行高效、准确的表示学习,但我们发现,当场景规模逐渐从物体、室内场景基本扩展到室外自动驾驶大场景下时,使用单一的多平面场景表示反而会产生退化的效果。在本文中,我们将提出一种混合场景表示方法,使得即便在更大规模的复杂场景中,依然可以用较小的存储空间获得准确的场景表示。

\newpage
\section{真实感的神经渲染方法:场景扭曲变换}
\label{sec: related-work realistic rendering}
除了使用不同的场景表示方法外,为了实现更真实的渲染效果,研究者还在其他方面做出了改进。其中,最为典型的方法是将三维欧式空间通过变换,映射到扭曲后的参数空间。

通常来讲,在场景图片数据集中,场景的内容在三维空间上并不是均匀分布的,如在前向观察的数据集中,通常场景内容随着深度的增加而反比例下降;在无边界、物体中心的数据集中,场景内容随着空间点到物体中心距离的增加而下降。因此,若能将在原始三维欧式空间中非均匀分布的空间点映射到一个均匀分布,则将能大大提升场景表示的利用率,从而提高渲染真实感。

\subsection{位置编码}
\label{sec: related-work positional encoding}

尽管神经网络被证明为通用的函数逼近器\cite{hornik_multilayer_1989},但研究者发现让网络直接在 输入坐标$(x,y,z,\theta,\phi)$上优化会导致渲染颜色。机器学习领域相关工作\cite{rahaman_spectral_2019}表明多层感知机偏向于学习低频函数,在高频信息上的学习速率显著低于低频信息的学习速率。而在将输入传递到网络之前,使用高频函数将输入映射到更高维空间可以更好地拟合包含高频变化的数据。

位置编码是一种常用于自注意力机制的技术,它通过将空间位置信息编码为向量来提高神经网络的表现力。但是Transformers\cite{vaswani_attention_2017}等基于注意力机制的方法为Token在整个序列中的离散位置提供隐式的位置坐标,然而在NeRF\cite{mildenhall_nerf_2020}中,通过将场景中的位置信息编码为含有位置信息的高维向量,可以将输入的xyz坐标映射到高维空间中,从而让网络学习场景中的高频细节信息,实现更真实的渲染效果。NeRF位置编码将输入数据$p$通过式\ref{eq: related-work PE}所示的方法映射到高维频域空间。
\begin{equation}
    \gamma(p) = \left(\sin(2^0\pi p),\cos(2^0\pi p),\cdots \sin(2^{L-1}\pi p),\cos(2^{L-1}\pi p)\right)
    \label{eq: related-work PE}
\end{equation}

\begin{figure}[ht]
    \centering
    \includegraphics[width=\textwidth]{undergraduate-thesis/images/related-work/NeRF ablation-PE.pdf}
    \caption{位置编码对NeRF网络性能的影响。}
    \label{fig:related-work nerf-PE-ablation}
\end{figure}

图\ref{fig:related-work nerf-PE-ablation}中展示了去掉位置编码后对NeRF网络渲染图片的效果影响。由图可以看出,去除高频位置信息后,NeRF网络很难表示图片中的高维几何细节,从而表现出“模糊”的特点。

从本质上,NeRF的位置编码实质上也是一个扭曲变换,将信息分布不均匀的空域映射到更加均匀的频域空间。

\subsection{基于Mip的反走样神经渲染}
尽管神经辐射场\cite{mildenhall_nerf_2020}及其变体\cite{muller_instant_2022,martin-brualla_nerf_2021,zhang_nerf_2020}在视图合成任务中展示了令人印象深刻的结果,但 NeRF 的渲染模型存在缺陷:当训练图像以多种分辨率观察场景内容,或分别从远处和近处观察同一位置时,NeRF的渲染在近景视图中显得过于模糊,并且在远景视图中包含混叠伪影。其原因在于NeRF将一个像素抽象为一个无穷小的没有面积的三维点,然而在真实世界中,像素在空间上占有一定的面积,从而相机发出的“光线”实际上应该表示为一个圆锥体(如图\ref{fig:related-work mip-nerf cone tracing}所示)。式\ref{eq: related-work volume rendering}中的NeRF渲染公式仅在一条射线上积分所有的辐射值,使用线积分的结果代替了视锥中全部三维点的重积分:
\begin{equation}
    c = \iiint_{\mathbf{p}\in V}\sigma(\mathbf{p})T(\mathbf{p})c(\mathbf{p},\mathbf{d})\text{d}p,
\end{equation}
其中$V$为相机原点$\mathbf{o}$朝方向$\mathbf{d}$发射的视锥,$p$为$V$中点。

\begin{figure}[h]
    \centering
    \includegraphics[width=0.8\textwidth]{undergraduate-thesis/images/related-work/mipnerf-intersection.png}
    \caption{NeRF方法产生走样原因的示意图。}
    \label{fig:related-work mip-nerf intersection}
\end{figure}

这样的简化必然会导致渲染误差的存在,进而产生走样的渲染结果。如图\ref{fig:related-work mip-nerf intersection}所示,当我们分别从两个不同远近、不同角度的视角同时观察到空间中的某个三维点。NeRF沿着每个像素的光线提取点采样位置编码特征,这些点采样特征忽略了每条射线观察到的体积的形状和大小,因此两个不同的相机以不同的比例对同一位置进行成像可能会产生相同的模糊点采样特征,从而显着降低 NeRF 的性能。

\begin{figure}[h]
    \centering
    \includegraphics[width=\textwidth]{undergraduate-thesis/images/related-work/mipnerf-cone.png}
    \caption{NeRF (a) 沿着从相机投影中心通过每个像素的光线进行采样,然后使用位置编码 (PE) $\gamma$ 对这些点进行编码以产生特征$\gamma(x)$。 Mip-NeRF (b) 对由相机像素定义的 3D 圆锥台进行推理,使用集成位置编码 (IPE) 对这些锥形截锥体进行特征化,使用多元高斯近似截锥体,然后计算积分 $E[\gamma(x)]$以近似超采样的效果。}
    \label{fig:related-work mip-nerf cone tracing}
\end{figure}

一个直接的解决方案是采用离线光线追踪中使用的策略:通过发射多条光线并对每个像素进行超采样。但这对于像 NeRF 这样的神经体积表示来说是非常昂贵的,它需要数百次查询多层感知机网络来渲染一条射线,并需要几个小时甚至几天来重建一个场景。Mip-NeRF \cite{barron_mip-nerf_2021}提出将采样过程显式建模为投射锥体而不是射线的过程,并对每个采样的锥台使用一个各向异性的高斯椭球表示。为了模拟对整个锥台进行超采样,Mip-NeRF提出用高斯椭球上位置编码的均值近似每个锥台内体积的位置编码的平均,即集成位置编码(Integrated Positional Encoding,IPE)。
\begin{figure}[h]
    \centering
    \includegraphics[width=\textwidth]{undergraduate-thesis/images/related-work/mipnerf-encoding.png}
    \caption{NeRF(左)和集成位置编码(IPE)(右)的一维可视化。由于 NeRF 沿每条射线对点进行采样并对所有频率进行同等编码,因此高频 PE 特征混叠,从而导致渲染伪影。通过在每个区间上使用集成位置编码特征,当频率周期与被集成的间隔大小相近时,IPE特征的高频维度缩小到零,从而产生隐式编码大小的抗锯齿特征。}
    \label{fig:related-work mip-nerf encoding}
\end{figure}

具体来讲,高斯椭球$(\mu,\Sigma)$内的集成位置编码可以写作:
\begin{align}
    \gamma(\mu,\Sigma) &= \text{E}_{\mathbf{x}\sim\mathcal{N}(\mu_\gamma,\Sigma_\gamma)}[\gamma(\mathbf{x})] \\
    &= \begin{bmatrix}
    \sin(\mu_\gamma)\circ\exp(-\frac{1}{2}\text{diag}(\Sigma_\gamma))\\
    \cos(\mu_\gamma)\circ\exp(-\frac{1}{2}\text{diag}(\Sigma_\gamma))
    \end{bmatrix}
\end{align}

由图\ref{fig:related-work mip-nerf encoding},IPE 特征的行为很直观:如果位置编码中的特定频率的周期大于用于构造 IPE 特征的区间宽度,则该频率的编码不受影响。但是,如果周期小于间隔,则该频率的编码将按比例缩小至零。简而言之,IPE 保留了在一个区间内恒定的频率并温和地“移除”在一个区间内变化的频率,而 PE 保留了直到某个手动调整的超参数 L的所有频率。通过以这种方式缩放每个三角函数值,IPE可以有效地抗锯齿位置编码功能,可以平滑地编码空间体积的大小和形状。

\begin{figure}[h]
    \centering
    \includegraphics[width=0.8\textwidth]{undergraduate-thesis/images/related-work/mipnerf-multi-scale.png}
    \caption{Mip-NeRF 和基线模型在不同图片分辨率上渲染的效果对比。}
    \label{fig:related-work mip-nerf results}
\end{figure}

图\ref{fig:related-work mip-nerf results}展示了Mip-NeRF使用IPE后与其他模型在渲染低分辨率图像时的效果对比。在全分辨率图像上训练的 NeRF 能够在新的视图位置生成逼真的渲染图,但仅限于训练图像的分辨率或比例;将相机向后拉并放大(或类似地,调整相机内在特性以降低图像分辨率)会导致呈现出严重的锯齿现象;在多分辨率图像上训练 NeRF 可以稍微改善这个问题,但会导致跨尺度渲染质量差:全分辨率模糊,低分辨率出现“锯齿”; Mip-NeRF也在多分辨率图像上训练,但能够在不同尺度上生成逼真的渲染。

\subsection{无界场景表示}
相比原始的位置编码和Mip-NeRF中的集成位置编码,Mip-NeRF中的无界场景映射则是一种更加直观的扭曲映射方法。

由于透视投影,远离相机放置的物体将占据图像平面的一小部分,但如果放置在附近,则会占据更多图像并且细节可见。因此,理想的 3D 场景参数化应该为附近的内容分配更多的容量,为远处的内容分配更少的容量。

最初的 NeRF 论文侧重于从360$^\circ$捕获具有蒙版背景的对象以及所有图像面向大致相同方向的正面场景(也称作Forward-Facing)。对于蒙面物体,NeRF 直接参数化 3D 欧几里得空间中的场景,但对于正面场景,NeRF 使用投影空间中定义的坐标(归一化设备坐标,或“NDC”空间)。通过将无限深的相机视锥变形为有界立方体,其中沿 z 轴的距离对应于视差(反距离),NDC 以与透视投影的几何形状一致的方式有效地重新分配 NeRF MLP 的容量。然而,在所有方向上都是无界的场景,而不仅仅是在一个方向上,需要不同的参数化。 不同于NeRF++\cite{zhang_nerf_2020}使用额外的网络对远处的物体进行建模,在Mip-NeRF 360\cite{barron_mip-nerf_2022}中,将这个想法扩展到 Mip-NeRF 并提出了一种将任何平滑参数化应用于体积的无界场景参数化方法。

\begin{figure}[ht]
    \centering
    \includegraphics[width=\textwidth]{undergraduate-thesis/images/mipnerf360-contraction.png}
    \caption{Mip-NeRF 360的无界场景压缩的二维示意图}
    \label{fig:related-work-unbounded-contraction}
\end{figure}

如图\ref{fig:related-work-unbounded-contraction}所示,对于任意一个输入坐标$x\in\mathbb{R}^3$,Mip-NeRF 360将输入坐标通过下面的公式映射到一个半径为2的球面内:
\begin{equation}
    \mathtt{contract}(\mathbf{x}) = \left\{\begin{matrix}&\mathbf{x}&||\mathbf{x}||<1\\&(2-\frac{1}{||\mathbf{x}||})(\frac{\mathbf{x}}{||\mathbf{x}||})&||\mathbf{x}||\geq1\end{matrix}\right.
\end{equation}

\subsection{小结}
在本节中,我们介绍了使用场景扭曲变换提高场景表示模型利用率的几种方法,然而目前的扭曲变换方法无法解决在更大规模的室内、室外复杂场景的高效表达,因而限制了神经隐式场在自动驾驶等领域的发展。在本文中我们提出基于神经点云的场景扭曲变换,使得场景表示可以自适应地拟合大规模场景。


\newpage
\section{辐射-距离混合隐式场}
\label{sec: related-work density-distance fields}
在本章的前两节中,我们介绍了以神经辐射场的神经场景表示的发展,然而单一的辐射场景表示对于场景几何的建模能力较差。在本节中,我们将首先分析辐射场这一不足的原因,接着分别从采样策略、体渲染模型、多元传感信息融合和混合隐式场优化的几个方面介绍辐射-距离混合隐式场。

\subsection{神经辐射场的形状-辐射二义性}
在没有正则化的情况下,NeRF\cite{mildenhall_nerf_2020}对仅依赖于视图的外观进行建模的能力导致 3D 形状和辐射之间存在一个二义性,使得网络可以接受退化的内在形状,而渲染出真实感颜色。对于任意的、不正确的形状,可以证明存在一组辐射场可以完美地解释训练图像,但不能很好地推广到新的测试视图。

为了说明这种歧义,假设对于给定的场景,我们将几何体表示为一个单位球体。换句话说,让我们将 NeRF 的不透明度场固定为在单位球体表面为 1,在其他地方为 0。然后,对于每个训练图像中的每个像素,我们将穿过该像素的光线与球体相交,并将交点处(以及沿光线方向)的辐射值定义为该像素的颜色。这个人工构造的解决方案是一个有效的 NeRF 重建,它完美地适合输入图像。

\begin{figure}[h]
    \centering
    \includegraphics[width=0.6\textwidth]{undergraduate-thesis/images/shape-radiance ambiguity.png}
    \caption{辐射-距离二义性示意图}
    \label{fig:related-work shape-radiance ambiguity}
\end{figure}

如图\ref{fig:related-work shape-radiance ambiguity}所示,假设一个二维曲面$S^*$的形状为图中所示,相机$C_0, C_1$分别从不同角度观察曲面上一点,由图可以看出,如果网络实际学到的几何形状为图中$\hat{S}$所代表的单位圆,而将$C_0, C_1$观察到的颜色“贴”在单位圆上。这样即使网络学习到的几何与真实几何形状完全不同,也可以渲染出真实感的颜色。这种二义性被叫做“形状-辐射二义性”\cite{zhang_nerf_2020},因而由辐射场建模的场景几何通常被认为是欠约束的。

基于这一观察,近年来,研究者提出同时优化辐射和距离场,并在体渲染过程中将这两个完全不同的隐式场的输出进行融合\cite{oechsle_unisurf_2021,gropp_implicit_2020,yariv_multiview_2020,yariv_volume_2021,wang_neus_2021,shao_doublefield_2022,darmon_improving_2022,ueda_neural_2022,long_sparseneus_2022,yu_monosdf_2022,wang_pet-neus_2023,yuan_monocular_2023,liang_hr-neus_2023,chen_dehazenerf_2023,zhu_vdn-nerf_2023,azinovic_neural_2022, sun_neural_2022}。 简单来说,就是将神经符号距离场(Neural SDF)通过几何变换转换为辐射场中体密度的过程。然而,由于这两种隐式场天然的差距,这一融合并不是一件易事。为了弥补隐式场直接的领域差距,研究者在采样策略\cite{yariv_volume_2021, mildenhall_nerf_2020, barron_mip-nerf_2022, oechsle_unisurf_2021},体渲染方法\cite{oechsle_unisurf_2021,yariv_volume_2021,wang_neus_2021},融合多元传感信息\cite{azinovic_neural_2022,yu_monosdf_2022}和隐式场优化进行了大量的研究工作,如图\ref{fig:related-work sdfstudio}所示\cite{yu_sdfstudio_2022}。

\begin{figure}[t]
    \centering
    \includegraphics[width=\textwidth]{undergraduate-thesis/images/sdfstudio.png}
    \caption{混合隐式场中各个组件的相关研究工作}
    \label{fig:related-work sdfstudio}
\end{figure}

\subsection{采样策略}
\subsubsection{分层采样}
\label{sec: coarse-to-fine sampling}
NeRF
\subsubsection{隐式表面引导的采样}

\subsubsection{基于误差的采样}


\subsubsection{基于Proposal网络的采样}
MipNeRF 360

\subsubsection{基于Occupancy的网格采样}

\subsection{体渲染方法}
\subsubsection{基于辐射场的体渲染方法}

\subsubsection{以UniSuRF为代表的体渲染方法}

\subsubsection{以VolSDF为代表的体渲染方法}

\subsubsection{以NeuS为代表的体渲染方法}

\subsection{多元传感信息融合}

\subsubsection{基于RGB-D的混合隐式场景表征学习}

\subsubsection{多元传感融合场景表示学习中的多视角一致性误差}

\subsubsection{多元传感融合场景表示学习中的渲染误差}

\subsubsection{基于全方向距离场的无偏场景表示学习}

\subsection{优化混合隐式场景表示}

\subsubsection{基于光度误差的辐射场优化}

...

\section{本章小结}